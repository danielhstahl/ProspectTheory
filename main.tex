\documentclass{article}
\usepackage{amsmath}
\usepackage{fancyhdr}
\usepackage{amsthm}
\usepackage{caption}
\usepackage{amsfonts}
\usepackage{graphicx}
\usepackage{cite}
\usepackage{float}
\usepackage{blindtext}
\setlength{\parindent}{0cm}
\newtheorem{theorem}{Theorem}
\newtheorem{corollary}{Corollary}
\DeclareMathOperator{\diag}{diag}
\newcommand{\norm}[1]{\left\lVert#1\right\rVert}

\title{Existence of Expectation Representation of Preferences under Uncertainty}
\date{}
\author{Daniel H. Stahl}

\begin{document}
\maketitle
\section{Introduction}

Expected Utility Theory (EUT) is the dominant economic paradigm for modeling preferences under uncertainty. Classic EUT is justified by the von Nuemann-Morgenstern (1947) theorem, which provides four axioms under which an expected utility representation exists to express preferences under uncertainty.  In 1953, Allais (1953) challenged this theorem through a thought experiment (the ``Allais Paradox'') where most individuals would prefer gambles that violated EUT.  Subsequent empirical work (for example, List (2005)) provides additional evidence that classic EUT is not sufficiently rich to capture human behavior under uncertainty. Prospect Theory, introduced by Kahneman and Tversky (1979) has offered an alternative to Expected Utility Theory (EUT) and addressed these empirical discrepancies in classic EUT. Kahneman and Tversky particularly challenged the ``independence'' or ``substitution'' axiom of the von Nuemann-Morgenstern proof.  Prospect Theory posits that individuals evaluate outcomes relative to a reference point and weigh probabilities non-linearly. The outcomes are represented by a (not-everywhere-differentiable) ``value'' function instead of a utility function, and the output of the value function in each state is multiplied by a function of the probability of that state occurring, successfully explaining anomalies such as the Allais Paradox.  
\\
\\
Kahneman and Tversky's paper sparked additional research and interest in modeling behavior under uncertainty without relying on EUT.  Quiggin (1982) introduced rank-dependent weightings on the probabilities; representing utility as the sum of the utility in each potential outcome weighted by a function of the (cumulative) probability of each outcome. Tversky and Kahneman (1992) incorporated Quiggin's approach, extending Prospect Theory into Cumulative Prospect Theory (CPT).  Research by Wakker and Tversky (1993) further refined the theoretical underpinnings of CPT and contrasted it with EUT.
\\
\\
While this literature builds a mathematically rigorous alternative to EUT, financial and economic literature continues to be dominated by EUT.  Notable exceptions include Barberis and Huang (2008) for financial asset pricing, Schmidt and Zank (2008) for analyzing risk-aversion with Prospect Theory, and Gonzalez and Wu (1999) and Prelec (1998) for statistically estimating the cumulative weighting function.  
\\
\\
In this paper, I prove the existence of an expectation representation of preferences without requiring the independence assumption of von Nuemann and Morgenstern.  I show that a measure exists such that preferences which violate the independence axiom can still be represented by an expectation.  This approach has two major benefits over the CPT.  First, the approach is a parsimonious extension of EUT: while a new measure does ``re-weight'' probabilities as in CPT, it does not require rank-ordering preferences, accumulating probabilities, or using a non-differentiable value function.  Second, by allowing an expectation representation, the vast literature built on EUT remains sound despite Kahneman and Tversky's objections to the independence axiom. I give a practical example by showing that preferences consistent with Allais' paradox can be represented by expectations.  

\section{The von Nuemann-Morgenstern Axioms}

Let \(X\) represent the consumption set available to economic agents.  A bundle \(x \in X\) is a set of consumption options within \(X\).  A gamble \(\sigma\) is a random variable that chooses (potentially) different consumption bundles within \(X\) for various states of the world.  In the language of probability theory, \(\sigma: \Omega \to X\).  I allow gambles to also exist in the consumption set, allowing me to state that \(\sigma \in X\).  Unless otherwise stated, I denote ``deterministic'' consumption bundles as \(x_i \in X\) while I denote gambles as \(\sigma_i \in X\). 
\\
\\
The von Nuemann-Morgenstern theorem states that given a set of axioms, there exists a function \(u: X \to \mathbf{R}\) such that for all \(\sigma_1, \sigma_2 \in X\), \(\mathbb{E}\left[u(\sigma_1)\right] \geq \mathbb{E}\left[u(\sigma_2)\right]\) if and only if \(\sigma_1 \succeq  \sigma_2\).  These axioms are:

\begin{enumerate}
	\item \textbf{Completeness}.  \(\forall x_1, \, x_2 \in X\), either \(x_1 \succeq  x_2\), \(x_2 \succeq  x_1\), or both.
	\item \textbf{Transitivity}.  \(\forall x_1, \, x_2,\, x_3 \in X\), if \(x_1 \succeq  x_2\) and \(x_2 \succeq  x_3\), then \(x_1 \succeq  x_3\).
	\item \textbf{Continuity}.  \(\forall x_1, \, x_2,\,x_3 \in X\) satisfying \(x_1 \succeq  x_2 \succeq  x_3\), there exists \(\alpha \in [0,\,1]\) such that \(\alpha x_1+(1-\alpha) x_3 \sim x_2\).
	\item \textbf{Independence}.  \(\forall x_1, \, x_2,\,x_3 \in X\) and \(\alpha \in [0,\,1]\), \(x_1 \succeq  x_2 \Leftrightarrow \alpha x_1+(1-\alpha)x_3 \succeq  \alpha x_2 +(1-\alpha)x_3\)
\end{enumerate}

Only the first two axioms are required for rationality.  The third is required to ensure the existence of a utility function (agents may be rational even if no utility function exists).  The last axiom, per von Nuemann-Morgenstern's proof, is required to ensure that preferences in an uncertain world can be represented by taking expectations. 
\\
\\
This theorem is powerful: it provides theoretical justification for using the mathematical language of expected utility as a toolkit for explaining human preferences and behavior.  However, the proof relies heavily on all four axioms.  If one of the axioms is shown to be false, the proof fails.  
\\
\\
The proof is constructive, demonstrating how a utility function can be created.  The construction relies on all four axioms.  However, utility functions may still exist if the third and fourth axioms are relaxed.  They simply cannot be constructed in the same way as in the theorem.  In other words, if the independence axiom is not valid, expected utility under the real-world measure \emph{may} still represent preferences, but it is not \emph{guaranteed} to do so.  

\section{The Allais Paradox} 

In 1953, Allais proposed the following two sets of choices of gambles:

\subsubsection{Choice 1}

\begin{equation*}
\left\{
\begin{array}{rl}
1\text{M} & \text{with probability } 1
\end{array} \right.
\end{equation*}

OR

\begin{equation*}
\left\{
\begin{array}{rl}
0 & \text{with probability } 0.01,
\\1\text{M} & \text{with probability } 0.89,
\\5\text{M} & \text{with probability } 0.1.
\end{array} \right.
\end{equation*}

\subsubsection{Choice 2}

\begin{equation*}
\left\{
\begin{array}{rl}
\\0\text{M} & \text{with probability } 0.89,
\\1\text{M} & \text{with probability } 0.11.
\end{array} \right.
\end{equation*}

OR

\begin{equation*}
\left\{
\begin{array}{rl}
0 & \text{with probability } 0.9,
\\5\text{M} & \text{with probability } 0.1.
\end{array} \right.
\end{equation*}

If the von Nuemann-Morgenstern theorem holds (using both the existence of a utility function and the independence axiom), the second choice can be decomposed as 
\begin{align*}
0.89u(0)+0.11u(1)\, &|\, 0.9u(0)+0.1u(5) \\
\Leftrightarrow u(1)-0.89u(1) \, &|\, 0.01u(0)+0.1u(5) \\
\Leftrightarrow u(1)-0.89u(1) \, &|\, 0.01u(0)+0.1u(5) \\
\Leftrightarrow u(1) \, &|\, 0.01u(0)+0.1u(5)+0.89u(1)
\end{align*}

This final statement is simply the utility of the first choice.  Hence if an agent chooses option one in choice one, he or she should choose option one in choice two.  Likewise, if the agent prefers option two in choice one, he or she should choose option two in choice two.
\\
\\
In reality, many people would prefer the first option for choice one and the second option for choice two.  Even the remote possibility of missing out on 1 million dollars is too big a gamble. For the remainder of this paper the independence axiom is dropped.  I show that even without this axiom an expectation representation exists for preferences under uncertainty.

 
\section{Existence of Utility Function and Probability Measure}

\begin{theorem} \label{theorem1}
Assume the first three von Nuemann-Morgenstern axioms, which implies the existence of an (ordinal) utility function \(g\) representing preferences such that \(g(x_1)\geq g(x_2) \Leftrightarrow x_1 \succeq x_2 \).  Assume a probability space which contains the outcomes of all relevant gambles. Then there exists a function \(u: X \to \mathbb{R}\) and a probability measure \(\mathbb{Q}\) equivalent to the ``physical'' measure \(\mathbb{P}\) such that \(\mathbb{E}_Q\left[u(\sigma_1)\right] \geq \mathbb{E}_Q\left[u(\sigma_2)\right]\) if and only if \(\sigma_1 \succeq \sigma_2\).  
\end{theorem} 

\begin{proof}
Debreu's Representation Theorem axioms 1, 2, and 3 are sufficient and necessary for the existence of the ordinal utility function \(g\).  I can denote the real-world expectation of the two gambles as \(\mathbb{E}[g(\sigma_1)]\) and \(\mathbb{E}[g(\sigma_2)]\). 

\subsubsection*{Case 1}

If \(\mathbb{E}[g(\sigma_1)]\geq\mathbb{E}[g(\sigma_2)]\) and \(\sigma_1 \succeq \sigma_2\), then \(\mathbb{Q}\) can be chosen to be \(\mathbb{P}\) and \(u(\cdot)\) can be chosen to be \(g(\cdot)\); and the work is done. Note that this will be the case under any measure if the outcome of \(\sigma_2\) is preferred to the outcome of \(\sigma_1\) in all states.

\subsubsection*{Case 2}

The interesting case is when \(\mathbb{E}[g(\sigma_1)]<\mathbb{E}[g(\sigma_2)]\) while \(\sigma_1 \succeq \sigma_2\).  Note that since \(g\) is ordinal I can always choose \(u=a+bg\) such that \(u\) is positive.  This lets me define a new probability measure:

\[\mathbb{Q}(A)=\frac{1}{\alpha}\int_A \frac{u(\sigma_1(\omega))}{u^2(\sigma_2(\omega))} \frac{dv(\omega) }{ d\mu(\omega)} d\mu(\omega)\] 
Where \(\alpha=\int_\Omega \frac{u(\sigma_1(\omega))}{u^2(\sigma_2(\omega))} dv(\omega) \), \(\mu\) is the real-world probability measure, and \(v\) is a ``free'' measure that agrees with the real-world measure on which events have measure zero.  Armed with this new measure, we can compute the new expectations:

\[\mathbb{E}_Q[u(\sigma_1)]=\frac{1}{\alpha} \int_\Omega \frac{u^2(\sigma_1(\omega))}{u^2(\sigma_2(\omega))} dv(\omega)\]
\[\mathbb{E}_Q[u(\sigma_2)]=\frac{1}{\alpha} \int_\Omega \frac{u(\sigma_1(\omega))}{u(\sigma_2(\omega))} dv(\omega)\]

Since \(\sigma_1\) is preferred to \(\sigma_2\) in at least some states of the world (or else we would be in Case 1), \(v\) can be chosen so that \(\int_\Omega \frac{u(\sigma_1(\omega))}{u(\sigma_2(\omega))} dv(\omega)\geq 1\).  For example, \(dv\) could simply be \(\mathbb{I}_{u(\sigma_1)>u(\sigma_2)}\).  Then by Jenson's inequality, 
\[\mathbb{E}_Q[u(\sigma_1)] \geq   \mathbb{E}_Q[u(\sigma_2)]^2 \geq  \mathbb{E}_Q[u(\sigma_2)] \implies \mathbb{E}_Q[u(\sigma_1)] \geq \mathbb{E}_Q[u(\sigma_2)] \]

\end{proof}

\begin{corollary}\label{corr1}
	If gambles \(\sigma_3\) and \(\sigma_4\) are independent of gambles \(\sigma_1\) and \(\sigma_2\) under measure \(v\), \(\mathbb{E}[u(\sigma_1)] < \mathbb{E}[u(\sigma_2)]\), and \(\mathbb{E}[u(\sigma_3)] < \mathbb{E}[u(\sigma_4)]\), then under the measure defined as 
	\[\mathbb{Q}(A)=\frac{1}{\alpha}\int_A \frac{u(\sigma_1(\omega))u(\sigma_3(\omega))}{u^2(\sigma_2(\omega))u^2(\sigma_4(\omega))} \frac{dv(\omega) }{ d\mu(\omega)} d\mu(\omega)\]
	\(\mathbb{E}_Q[u(\sigma_1)]\geq\mathbb{E}_Q[u(\sigma_2)]\) and \(\mathbb{E}_Q[u(\sigma_3)]\geq\mathbb{E}_Q[u(\sigma_4)]\) where 
	\\
	\(\alpha=\int_\Omega \frac{u(\sigma_1(\omega))u(\sigma_3(\omega))}{u^2(\sigma_2(\omega))u^2(\sigma_4(\omega))} dv(\omega) \).
\end{corollary}

\begin{proof}
This follows from Theorem \ref{theorem1} and by the fact that \(\mathbb{E}[XY]=\mathbb{E}[X]\mathbb{E}[Y]\) if \(X\) and \(Y\) are independent.  
\end{proof}

\section{Selecting a Probability Measure}

The previous section shows that a measure could be constructed.  This section proposes a method to construct the measure in a discrete setting.

Let there be \(n\) states of the world, with probability of each state given by the vector 
\[p = [p_1, p_2, \dotsc, p_n]\]  
Let there be \(m\) gambles represented by the matrix 
\[\Sigma = \begin{bmatrix}
    \sigma_{11}       & \sigma_{12} & \dots & \sigma_{1n} \\
    \sigma_{21}       & \sigma_{22}  & \dots & \sigma_{2n} \\
    \hdotsfor{4} \\
    \sigma_{m1}       & \sigma_{m2} & \dots & \sigma_{mn}
\end{bmatrix}\]

Permissible gambles implies that the rank of \(\Sigma\) is at most \(n\); that is, if there are more gambles than states in the world then either there are equivalent gambles or there are gambles that are strictly dominated by other gambles and are not relevant for this exercise.  This is a similar argument to the no-arbitrage condition in financial markets.  
\\
\\
Parsimony suggests identifying a measure \(q\) so that \(\mathbb{Q}\) and \(\mathbb{P}\) are ``close'' to each other.  This motivates a constrained least squares problem:

\begin{equation}
\begin{aligned}
\min_v \quad \norm{q-p} \\
\textrm{s.t.}\quad u(\Sigma) q = z\\
q \geq 0
\end{aligned}
\end{equation}

Where \(z\) is the ``actual'' utility and whose ordering of the outcomes under \(\mathbb{Q}\) aligns with the ordering of preferences.  This problem can be solved using Quadratic Programming, and many programming languages have libraries capable of this optimization. For example, in the programming language Python, the ``SciPy'' optimization library can perform this type of constrained optimization.

\subsection{Examples}

\subsubsection{Allias Paradox choice 1}
Consider the first choice of the Allias Paradox.  The outcomes can be written in matrix form as follows (numbers in millions):

\[X^T=\begin{bmatrix}
	1 & 0  \\
	1 & 1  \\
	1 & 5  \\
\end{bmatrix}\]

Assuming a starting wealth of \(\$100,000\) and a log utility function, the (ordinal) utilities associated with these outcomes are:
\[M^T=\begin{bmatrix}
	13.911 & 11.513  \\
	13.911 & 13.911  \\
	13.911 & 15.445  \\
\end{bmatrix}\]

Since these values are positive, the function \(u\) can be set to the ordinal utility function \(g\).  The expected utility from these two outcomes under the real-world measure is 

\begin{center} 
	\begin{tabular}{c c}
		& Expected Utility under \(\mathbb{P}\) \\
		Choice 1 Option 1 & 13.911 \\
		Choice 1 Option 2 & 14.040 \\
	\end{tabular}
\end{center}

With a quadratic programming solver, a possible \(q\) is chosen as \(q(\omega)=[0.208, 0.735, 0.057]\).

Under this measure, the expected utility for each outcome is 

\begin{center} 
	\begin{tabular}{c c}
		& Expected Utility under \(\mathbb{Q}\) \\
		Choice 1 Option 1 & 13.911 \\
		Choice 1 Option 2 & 13.500 \\
	\end{tabular}
\end{center}

\subsubsection{Allias Paradox all choices}

The outcomes of all four choices can be represented as the following matrix (in millions):

\[X^T=\begin{bmatrix}
1 & 0 & 0 & 0 \\
1 & 1 & 0 & 0 \\
1 & 5 & 0 & 0 \\
1 & 0 & 1 & 0 \\
1 & 1 & 1 & 0 \\
1 & 5 & 1 & 0 \\
1 & 0 & 0 & 5 \\
1 & 1 & 0 & 5 \\
1 & 5 & 0 & 5 \\
1 & 0 & 1 & 5 \\
1 & 1 & 1 & 5 \\
1 & 5 & 1 & 5
\end{bmatrix}\]

Using the same utility function as in the first example,

\[M^T=\begin{bmatrix}
	13.911 & 11.513 & 11.513 & 11.513  \\
	13.911 & 13.911  & 11.513 & 11.513 \\
	13.911 & 15.445 & 11.513 & 11.513 \\
	13.911 & 11.513 & 13.911 & 11.513 \\
	13.911 & 13.911  & 13.911 & 11.513 \\
	13.911 & 15.445  & 13.911 & 11.513 \\
	13.911 & 11.513 & 11.513  & 15.445 \\
	13.911 & 13.911  & 11.513  & 15.445 \\
	13.911 & 15.445  & 11.513  & 15.445 \\
	13.911 & 11.513  & 13.911  & 15.445 \\
	13.911 & 13.911  & 13.911  & 15.445 \\
	13.911 & 15.445  & 13.911  & 15.445 \\
\end{bmatrix}\]


A possible \(q\) is chosen as 
\begin{equation*}
\begin{aligned}
q(\omega)=[0.19306, 0.64141, 0.11376, 0.00011, \\ 
0.00011, 0.00011, 0.05089, 0.00011, \\
0.00011, 0.00011, 0.00011, 0.00011]
\end{aligned}
\end{equation*}

Under this measure, the expected utility for each outcome is 

\begin{center} 
	\begin{tabular}{c c}
		& Expected Utility under \(\mathbb{Q}\) \\
		Choice 1 Option 1 & 13.911 \\
		Choice 1 Option 2 & 13.501 \\
		Choice 2 Option 1 & 11.515 \\
		Choice 2 Option 2 & 11.715 \\
	\end{tabular}
\end{center}

This resolves the paradox.  

\section{Conclusion}

This paper establishes that preferences can be represented by expectations even without relying on the independence axiom of von Neumann and Morgenstern. A measure exists such that preferences violating the independence axiom can still conform to an expectation representation. The proof is exemplified through preferences consistent with Allais' paradox, demonstrating that economic insights from EUT remain robust despite challenges posed by Kahneman and Tversky. This result suggests that the foundational principles of EUT can still apply to decision-making scenarios that deviate from classical assumptions.
\\
\\
The sole author states that there is no conflict of interest.
\\
\\
All code used in this paper can be found here: 
\\
https://github.com/danielhstahl/ProspectTheory.  
\\
\\
There is no data used in this paper.
\subsection*{Bibliography}
\begin{enumerate}
    \item Allais, M. (1953). Le comportement de l'homme rationnel devant le risque: critique des postulats et axiomes de l'école américaine. \textit{Econometrica}, 21(4), 503-546.
    \item Barberis, N., and Huang, M. (2008). Stocks as lotteries: The implications of probability weighting for security prices. \textit{American Economic Review}, 98(5), 2066-2100.
    \item Gonzalez, R., and Wu, G. (1999). On the shape of the probability weighting function. \textit{Cognitive Psychology}, 38(1), 129-166.
    \item Kahneman, D. and Tversky, A. (1979). Prospect theory: An analysis of decision under risk. \textit{Econometrica}, 47(2), 263-291.
    \item List, J. and Haigh, M. (2005).  A simple test of expected utility theory using professional traders.  List JA, Haigh MS. (2005). A simple test of expected utility theory using professional traders. \textit{Proc Natl Acad Sci USA}, 102(3), 945–948.
    \item Neumann, John von and Morgenstern, Oskar (1947). Theory of Games and Economic Behavior. Princeton University Press.
    \item Prelec, D. (1998). The probability weighting function. \textit{Econometrica}, 66(3), 497-527.
    \item Quiggin, J. (1982). A theory of anticipated utility. \textit{Journal of Economic Behavior and Organization}, 3(4), 323-343.
    \item Schmidt, U., and Zank, H. (2008). Risk aversion in cumulative prospect theory. \textit{Management Science}, 54(1), 208-216.
    \item Tversky, A., and Kahneman, D. (1992). Advances in prospect theory: Cumulative representation of uncertainty. \textit{Journal of Risk and Uncertainty}, 5(4), 297-323.
    \item Wakker, P., and Tversky, A. (1993). An axiomatization of cumulative prospect theory. \textit{Journal of Risk and Uncertainty}, 7(2), 147-175.
    
\end{enumerate}
	
\end{document}
