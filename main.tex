\documentclass{article}
\usepackage{amsmath}
\usepackage{fancyhdr}
\usepackage{amsthm}
\usepackage{caption}
\usepackage{amsfonts}
\usepackage{graphicx}
\usepackage{cite}
\usepackage{float}
\setlength{\parindent}{0cm}
\newtheorem{theorem}{Theorem}
\begin{document}
	
	
\title{Unified Framework of Rational Agents}
\date{}
\author{Daniel H. Stahl}

\section{Introduction}

The von Nuemann-Morgenstern theorem provides theoretical justification for the use of expected utility as a framework for analyzing decisions under uncertainty.  The axioms underlying the theorem have come under some scrutiny.  In psychology, Daniel Kahneman and Amos Tversky have introduced alternative representations of choice under uncertainty.  Kahneman and Tversky openly question expected utility in their papers (Kahneman and Tversky 1992).  Their primary objection is to the appropriateness of the independence axiom from the von Nuemann-Morgenstern proof.    
\\
\\
Kahneman is popular in both academic and popular press for his criticisms of traditional utility theory under uncertainty.  In his popular book ``Thinking, fast and slow'', Kahneman reiterates his belief that humans, unlike ``homo-economis'' are irrational.  The statement that humans are not rational is much stronger than the statements in his academic writings that humans do not seek to maximize their expected utility.  Indeed humans may be rational even if a von Nuemann-Morgenstern utility function does not exist.  
\\
\\
In this paper, I prove the existence of an expectation representation of preferences without requiring the independence assumption of von Nuemann and Morgenstern.  I show that a measure exists such that preferences which violate the independence axiom can still be represented by an expectation.  I give a practical example of this proof by showing that preferences consistent with Allais' paradox can be represented by expectations.  The conclusion is that economic insights generated by expected utility theory remain valid despite Kahneman's and Tversky's objections.   
\\
\\
Additionally, I clarify both what the von Nuemann-Morgenstern theorem actually states (and what it doesn't) as well as the implications of relaxing the independence assumption.


\section{The von Nuemann-Morgenstern Axioms}

Let \(X\) represent the consumption set available to economic agents.  A bundle \(x \in X\) is a set of consumption options within \(X\).  A gamble \(\sigma\) is a random variable that chooses (potentially) different consumption bundles within \(X\) for various states of the world.  In the language of probability theory, \(\sigma: \Omega \to X\).  I allow gambles to also exist in the consumption set, allowing me to state that \(\sigma \in X\).  Unless otherwise stated, I denote ``deterministic'' consumption bundles as \(x_i \in X\) while I denote gambles as \(\sigma_i \in X\).     
\\
\\
The von Nuemann-Morgenstern theorem states that given a set of axioms, there exists a function \(u: X \to \mathbf{R}\) such that for all \(\sigma_1, \sigma_2 \in X\), \(\mathbb{E}\left[u(\sigma_1)\right] \geq \mathbb{E}\left[u(\sigma_2)\right]\) if and only if \(\sigma_1 \succeq  \sigma_2\).  These axioms are:

\begin{enumerate}
	\item \textbf{Completeness}.  \(\forall x_1, \, x_2 \in X\), either \(x_1 \succeq  x_2\), \(x_2 \succeq  x_1\), or both.
	\item \textbf{Transitivity}.  \(\forall x_1, \, x_2,\, x_3 \in X\), if \(x_1 \succeq  x_2\) and \(x_2 \succeq  x_3\), then \(x_1 \succeq  x_3\).
	\item \textbf{Continuity}.  \(\forall x_1, \, x_2,\,x_3 \in X\) satisfying \(x_1 \succeq  x_2 \succeq  x_3\), there exists \(\alpha \in [0,\,1]\) such that \(\alpha x_1+(1-\alpha) x_3 \sim x_2\).
	\item \textbf{Independence}.  \(\forall x_1, \, x_2,\,x_3 \in X\) and \(\alpha \in [0,\,1]\), \(x_1 \succeq  x_2 \Leftrightarrow \alpha x_1+(1-\alpha)x_3 \succeq  \alpha x_2 +(1-\alpha)x_3\)
\end{enumerate}

Only the first two axioms are required for rationality.  The last two are only required in order to demonstrate the existence of a utility function.  This proves that humans may be rationale even if no such utility function exists.  
\\
\\
This theorem is powerful: it provides theoretical justification for using the mathematical language of expected utility as a toolkit for explaining human preferences and behavior.  However, the proof relies heavily on all four axioms.  If one of the axioms is shown to be false, the proof fails.  
\\
\\
It is important to clarify what the failure of the proof means for expected utility theory. The proof is constructive, demonstrating how a utility function can be created.  The construction relies on all four axioms.  However, utility functions may still exist if the third and fourth axioms are relaxed.  The functions cannot be constructed in the same way as in the theorem.  If the independence axiom is not valid, expected utility theory may still be valid.  However, it is not guaranteed to be valid.  
 

\section{The Allais Paradox} 

In 1953, Allais proposed the following two sets of choices of gambles:

\subsubsection{Choice 1}

\begin{equation*}
\left\{
\begin{array}{rl}
1\text{M} & \text{with probability } 1
\end{array} \right.
\end{equation*}

OR

\begin{equation*}
\left\{
\begin{array}{rl}
0 & \text{with probability } 0.01,
\\1\text{M} & \text{with probability } 0.89,
\\5\text{M} & \text{with probability } 0.1.
\end{array} \right.
\end{equation*}

\subsubsection{Choice 2}

\begin{equation*}
\left\{
\begin{array}{rl}
\\0\text{M} & \text{with probability } 0.89,
\\1\text{M} & \text{with probability } 0.11.
\end{array} \right.
\end{equation*}

OR

\begin{equation*}
\left\{
\begin{array}{rl}
0 & \text{with probability } 0.9,
\\5\text{M} & \text{with probability } 0.1.
\end{array} \right.
\end{equation*}

If the von Nuemann-Morgenstern theorem holds (using both the existence of a utility function and the independence axiom), the second choice can be decomposed as 
\begin{align*}
0.89u(0)+0.11u(1)\, &|\, 0.9u(0)+0.1u(5) \\
\Leftrightarrow u(1)-0.89u(1) \, &|\, 0.01u(0)+0.1u(5) \\
\Leftrightarrow u(1)-0.89u(1) \, &|\, 0.01u(0)+0.1u(5) \\
\Leftrightarrow u(1) \, &|\, 0.01u(0)+0.1u(5)+0.89u(1)
\end{align*}

This final statement is simply the utility of the first choice.  Hence if an agent chooses option one in choice one, he or she should choose option one in choice two.  Likewise, if the agent prefers option two in choice one, he or she should choose option two in choice two.
\\
\\
In reality, many people would prefer the first option for choice one and the second option for choice two.  Even the remote possibility of missing out on 1 million dollars is too big a gamble. 
\\
\\
Allais's paradox as well as subsequent empirical work by Kahneman and Tversky provides substantial evidence that the independence axiom may not be appropriate in many situations.  For the remainder of this paper the independence axiom is dropped.  

 
\section{Existence of Utility Function and Probability Measure}

\begin{theorem}
Assume axioms one through three hold.  Assume a finite probability space which contains the outcomes of all relevant gambles  and the existence of an ordering of preferences such that larger numbers are preferred to smaller numbers.  Assume that if a gamble has higher value in every state of the world than another gamble, that the first gamble is preferred.  Then there exists a function \(u: X \to \mathbb{R}\) and a probability measure \(\mathbb{Q}\) equivalent to the ``physical'' measure \(\mathbb{P}\) such that \(\mathbb{E}_Q\left[u(\sigma_1)\right] \geq \mathbb{E}_Q\left[u(\sigma_2)\right]\) if and only if \(\sigma_1 \succeq \sigma_2\).  
\end{theorem}

\begin{proof}
By Debreu's Representation Theorem, axioms 1, 2, and 3 are sufficient and necessary for there to exist an (ordinal) utility function \(g\) representing preferences such that \(g(x_1)\geq g(x_2) \Leftrightarrow x_1 \succeq x_2 \).  Let the realized utility from a set of \(m\) gambles be provided by the following matrix:

\[M=\begin{bmatrix}
	g(x_{1, 1}) & \ldots & g(x_{1, n})\\
	\vdots & \ddots & \vdots \\
	g(x_{m, 1}) & \ldots & g(x_{m, n})
\end{bmatrix}
\]

Where \(x_{i, j} \in X\) is the bundle from gamble \(i\) in state of the world \(j\).  Each row  \(i\) of the matrix represents the mapping of the results of the gamble \(\sigma_i\) to the consumption set. Let the preferences from each of these gambles be represented by a set of \(m\) numbers

\[z=\begin{bmatrix}
z_1 \\ \vdots\\ z_m
\end{bmatrix}\]


Note that these numbers are only intended to communicate relative preference.  In other words, \(z_i > z_k \Leftrightarrow \sigma_i \succeq \sigma_k\).  These numbers do not reflect expectations or the results of applying functions to the gambles.  Indeed, these numbers are identified as part of the proof.

MAYBEDELETEAdditionally, a negative \(z_i\) represents a gamble that leads to a decline in utility in every state.   
\\
\\
The proof is complete if I can show that there exists a \(q \in \mathbb{R}^n\) such that 
\begin{enumerate}
	\item \(Mq=z\)
	\item \(q_j>0 \, \forall q_j \in q\)
\end{enumerate}
Where \(z\) is chosen such that \(z_i \geq z_j\) if and only if  \(\sigma_i \succeq \sigma_j\).

If such a vector exists, then preferences can be represented as follows:

\[z_i=\sum_j q_j M_{i, j} = \mathbb{E}_Q \left[ c g(\sigma_i) \right] =\mathbb{E}_Q\left[u(\sigma_i)\right] \]

Where \(c=\sum_i q_i\) and \(u:X \to \mathbb{R}\) is defined as \( u(\cdot)=c g(\cdot)\). Since \(z_i > z_k \Leftrightarrow \sigma_i \succeq \sigma_k\), \(\mathbb{E}_Q\left[u(\sigma_1)\right] \geq \mathbb{E}_Q\left[u(\sigma_2)\right]\) if and only if \(\sigma_1 \succeq \sigma_2\).  
\\
\\
The remainder of this proof is demonstrating the existence of \(q\).  If the rank of \(M\) is equal to \(m\), there exists a \(q\) (not necessarily positive) which solves the equation \(Mq=z\) for every \(z\).  If the rank is less than \(m\), then for a solution to exist, the redundant row of \(M\) must have a solution in \(z\) that is consistent.  This is possible since we are free to choose \(z\); consistency in this case is implied by rationality. By this construction, there exists a \(q\) (not necessarily positive) that solves the equation \(Mq=z\).  
\\
\\
This is a frustrating proof.  Farka's Lemma doesn't help too much.  I merely need to show that there exists a positive vector \(q\) such that \(Mq\) can be ranked arbitrarily.  I can even make \(M\) all positive if need be.




\end{proof}


\section{Application to Allais Paradox}

The outcomes of the gambles in the Allais paradox can be written as the following matrix (in millions):

\[X^T=\begin{bmatrix}
1 & 0 & 0 & 0 \\
1 & 1 & 0 & 0 \\
1 & 5 & 0 & 0 \\
1 & 0 & 1 & 0 \\
1 & 1 & 1 & 0 \\
1 & 5 & 1 & 0 \\
1 & 0 & 0 & 5 \\
1 & 1 & 0 & 5 \\
1 & 5 & 0 & 5 \\
1 & 0 & 1 & 5 \\
1 & 1 & 1 & 5 \\
1 & 5 & 1 & 5
\end{bmatrix}\]

The probabilities associated with these outcomes are the following vector:

\[p=\begin{bmatrix}
0.008 \\
0.713 \\
0.080 \\
0.001 \\
0.088 \\
0.001 \\
0.001 \\
0.079 \\
0.009 \\
0.000 \\
0.010 \\
0.001
\end{bmatrix}\]

Assuming a starting wealth of \(\$100,000\) and a log utility function, the expected utility from each of the four possible demonstrate Allais paradox:


\begin{center} 
	\begin{tabular}{c c}
		& Expected Utility under \(\mathbb{P}\) \\
		Choice 1 Option 1 & 13.911 \\
		Choice 1 Option 2 & 14.040 \\
		Choice 2 Option 1 & 11.777 \\
		Choice 2 Option 2 & 11.906 \\
	\end{tabular}
\end{center}

Now consider the vector 

\[q=\begin{bmatrix}
0.04 \\
0.16 \\
0.02 \\
0.04 \\
0.07 \\
0.18 \\
0.01 \\
0.13 \\ 
0.07 \\
0.18 \\
0.03 \\
0.07
\end{bmatrix}\]

This vector sums to one and agrees with \(\mathbb{P}\) on the possible outcomes.  Hence \(q\) defines the probabilities for a probability measure \(\mathbb{Q}\) equivalent to \(\mathbb{P}\).  Under this measure, the utility of the choices are shown as follows:

\begin{center} 
	\begin{tabular}{c c}
		& Expected Utility under \(\mathbb{Q}\) \\
		Choice 1 Option 1 & 13.911 \\
		Choice 1 Option 2 & 13.785 \\
		Choice 2 Option 1 & 12.880 \\
		Choice 2 Option 2 & 13.440 
	\end{tabular}
\end{center}

Note that the choice of measure resolves the paradox.  



	
\end{document}